\documentclass[a4paper, 11pt, twoside]{article}

\usepackage[brazilian]{babel} % Habilita o uso de portugues brasileiro
\usepackage{lmodern}
\usepackage[T1]{fontenc}

\usepackage[square, numbers, comma, sort&compress]{natbib} 
\usepackage{verbatim}  % Comment environment
\usepackage[margin=2.5cm]{geometry}
\usepackage{amsmath}	% Matrices
\usepackage{subfig}
\usepackage{graphicx}
\usepackage{url}
\usepackage{hyperref}
\usepackage{xcolor}
\usepackage{algorithm}
\usepackage{algpseudocode}
\usepackage[retainorgcmds]{IEEEtrantools}


% from https://www.overleaf.com/learn/latex/code_listing
\usepackage{textcomp}  % required for upticks in listings env
\usepackage{listings}
\definecolor{codegreen}{rgb}{0, 0.6, 0}
\definecolor{codegray}{rgb}{0.5, 0.5, 0.5}
\definecolor{codepurple}{rgb}{0.58, 0, 0.82}
\definecolor{backcolour}{rgb}{0.95, 0.95, 0.95}

\lstset{backgroundcolor=\color{backcolour},   
	commentstyle=\color{codegreen},
	keywordstyle=\color{magenta},
	numberstyle=\tiny\color{codegray},
	stringstyle=\color{codepurple},
	basicstyle=\ttfamily\footnotesize,
	breakatwhitespace=false,         
	breaklines=true,                 
	captionpos=b,                    
	keepspaces=true,                                 
	showspaces=false,                
	showstringspaces=false,
	showtabs=false,                  
	tabsize=4,
	upquote=true,
	frame=single,
	language=Python
}

\hypersetup{urlcolor=blue, colorlinks=true, linkcolor=blue, citecolor=blue}  % Colours hyperlinks in blue, but this can be distracting if there are many links.

% Use double bar for vector norm
\providecommand{\norm}[1]{\lVert#1\rVert}

% "upright" 'd' for integrals
\newcommand{\ud}{\,\mathrm{d}}


\author{Fabio Casagrande Hirono \\ \texttt{fchirono@gmail.com}}
\title{\texttt{amiet\_tools}: um pacote em Python para predição de ruído de interação turbulência-aerofólio\\\textcolor{red}{EM CONSTRUÇÃO}}
\date{Janeiro 2021}

\begin{document}
\maketitle

\begin{abstract}
	O pacote \verb|amiet_tools| (AmT), escrito em linguagem Python, é uma implementação do modelo analítico de Amiet \cite{Amiet75} para predição de ruído de interação turbulência-aerofólio, com extensões. As funções permitem o cálculo do ``salto'' de pressão na superfície do aerofólio (i.e. a distribuição da fonte acústica) gerado em resposta à turbulência incidente, e do campo acústico radiado por esta interação. A turbulência incidente pode ser uma única rajada senoidal, ou uma soma de rajadas incoerentes com amplitudes definidas por um espectro energético. Outras funções includem o modelamento de efeitos de convecção e refração na propagação sonora para modelar medições acústicas realizadas em túneis de vento abertos ou fechados. Uma publicação de referência está disponível \cite{Casagrande_etal2020}, e o pacote pode ser encontrado através do link \url{https://github.com/fchirono/amiet_tools}.
\end{abstract}

% *-*-*-*-*-*-*-*-*-*-*-*-*-*-*-*-*-*-*-*-*-*-*-*-*-*-*-*-*-*-*-*-*-*-
\section{Introdução}

Ruído de aerofólio: importância, mecanismos físicos, modelo analítico.

Referências teóricas:

\begin{itemize}
	\item R. Amiet, ``\emph{Acoustic radiation from an airfoil in a turbulent stream}'', Journal of Sound and Vibration, Vol. 41, No. 4:407–420, 1975 \cite{Amiet75};
	
	\item G. Reboul, ``\emph{Modélisation du bruit à large bande de soufflante de turboréacteur}'', PhD Thesis, Laboratoire de Mécanique des Fluides et d’Acoustique - École Centrale de Lyon, Lyon - France, 2010 \cite{Reboul10};
	
	\item M. Roger, ``\emph{Broadband noise from lifting surfaces: Analytical modeling and experimental validation}'', in: R. Camussi (Ed.), ``\emph{Noise Sources in Turbulent Shear Flows: Fundamentals and Applications}'', Springer-Verlag, 2013 \cite{Roger13};
	
	\item L. de Santana, ``\emph{Semi-analytical methodologies for airfoil noise prediction}'', PhD Thesis, Faculty of Engineering Sciences - Katholieke Universiteit Leuven, Leuven, Belgium, 2015 \cite{deSantana2015};
	
	\item F. Casagrande Hirono, ``\emph{Far-Field Microphone Array Techniques for Acoustic Characterisation of Aerofoils}'', PhD Thesis, Institute of Sound and Vibration Research, University of Southampton, Southampton - UK, 2018 \cite{Casagrande18}.
\end{itemize}


% *-*-*-*-*-*-*-*-*-*-*-*-*-*-*-*-*-*-*-*-*-*-*-*-*-*-*-*-*-*-*-*-*-*-
\section{Usando o pacote}

\subsection{Requerimentos}

O pacote \verb|amiet_tools| (abreviado por ``AmT'') foi desenvolvido em Python 3.x, e tem como dependências os pacotes \verb|numpy| e \verb|scipy|. Para plotar resultados, esse tutorial usa o pacote \verb|matplotlib|, mas esta não é uma dependência do AmT. Estes pacotes estão inclusos na ``\emph{Anaconda Python Distribution}''\footnote{Disponível em \url{https://www.anaconda.com/products/individual}.}, uma distribuição Python gratuita e de código aberto. A distribuição Anaconda é usada para desenvolver o pacote AmT, e é a distribuição recomendada para usar AmT.

%Assim como outros programas de computação científica, o pacote AmT exige  certos recursos computacionais. Por exemplo, um aerofólio cuja superfície é discretizada com $(N_x, N_y)$ pontos terá a sua matriz de espectro cruzado (CSM) com $(N_x \times N_y)^2$ números complexos de precisão dupla (\verb|numpy.complex128|, 128 bits por número complexo), ocupando $(N_x \times N_y)^2 \times 128$ bits de memória RAM \emph{por frequência}. A amostragem padrão do pacote AmT, com $N_x = 100$ e $N_y = 101$, resultará em uma CSM que ocupa aproximadamente 1.6 GB por frequência. 

\subsection{Definindo variáveis com arquivos \texttt{.txt}}

O pacote AmT utiliza dois arquivos de texto contendo as variáveis relacionadas às condições do experimento e as variáveis relacionadas à geometria do aerofólio. Usaremos como exemplo os dois arquivos de configuração do teste ``DARP2016'', realizado no túnel de vento DARP da Universidade de Southampton, Reino Unido, e disponíveis no repositório GitHub do projeto.

O primeiro arquivo chama-se \verb|DARP2016_TestSetup.txt|, e os seus conteúdos estão indicados na listagem \ref{lst:TestSetup} abaixo; note que linhas vazias ou iniciadas com o caractere ``\#'' são ignoradas. O programa lerá os valores em cada linha do arquivo, e os interpretará conforme indicado nas respectivas linhas.

\begin{lstlisting}[caption={Arquivo \texttt{DARP2016\_TestSetup.txt}}, label={lst:TestSetup}]
# amiet_tools Test Setup file
#
# DARP2016 test setup, ISVR, Univ. of Southampton, UK

# Acoustic characteristics
340.	# c0		Speed of sound [m/s]
1.2		# rho0		Air density [kg/m**3]
20e-6	# p_ref		Ref acoustic pressure [Pa RMS]

# turbulent flow properties
60		# Ux				flow velocity [m/s]
0.025	# turb_intensity	turbulence intensity = u_rms/U
0.007	# length_scale		turb length scale [m]

# shear layer height
-0.075	# z_sl			Shear layer height (aerofoil is at z=0) [m]
\end{lstlisting}

O segundo arquivo chama-se \verb|DARP2016AirfoilGeom.txt|, e contém variáveis relacionadas à geometria do aerofólio. Seus conteúdos estão indicados na listagem \ref{lst:AirfoilGeom} abaixo.

\begin{lstlisting}[caption={Arquivo \texttt{DARP2016\_AirfoilGeom.txt}}, label={lst:AirfoilGeom}]
# amiet_tools AirfoilGeom file
#
# DARP2016 flat plate airfoil, ISVR, Univ. of Southampton, UK

0.075	# b			airfoil half chord [m]
0.225	# d 		airfoil half span [m]
100		# Nx 		number of chordwise points (non-uniform sampl)
101		# Ny 		number of spanwise points (uniform sampl)
\end{lstlisting}

Para carregar os valores descritos nos arquivos de texto, usa-se o programa descrito na listagem \ref{lst:LoadingVars}. Primeiramente, o pacote AmT é importado pelo script. A função \verb|AmT.loadTestSetup| é usada para carregar as variáveis relacionadas às condições do experimento a partir do arquivo de texto fornecido como argumento, e cria um objeto - aqui chamado de \verb|DARP2016Setup| - para armazenar as variáveis. Este objeto é uma instância da classe \verb|TestSetup|, definida no pacote AmT; mais informações sobre as classes usadas no pacote AmT podem ser encontradas na Seção \ref{sec:ClassDescription}.

Similarmente, usa-se a função \verb|AmT.loadAirfoilGeom| para carregar variáveis relacionadas à geometria do aerofólio a a partir de um arquivo de texto. Esta criará um objeto - aqui chamado de \verb|DARP2016Airfoil| - para armazenar as dimensões do aerofólio e as configurações de amostragem de sua superfície. Este objeto é uma instância da classe \verb|AirfoilGeom|, definida no pacote AmT; mais informações sobre as classes usadas no pacote AmT podem ser encontradas na Seção \ref{sec:ClassDescription}.

\begin{lstlisting}[caption={Importando pacotes e carregando variáveis},label={lst:LoadingVars}]
import amiet_tools as AmT

# load test setup from file
DARP2016Setup = AmT.loadTestSetup('../DARP2016_TestSetup.txt')

# load airfoil geometry from file
DARP2016Airfoil = AmT.loadAirfoilGeom('../DARP2016_AirfoilGeom.txt')
\end{lstlisting}

Alternativamente, estas funções podem ser chamadas sem nenhum argumento de entrada; o objeto resultante irá conter as configurações do teste ``DARP2016'' por padrão, e estes podem ser modificados posteriormente dentro do script. Um exemplo deste método alternativo encontra-se na listagem \ref{lst:AltLoadingVars}, onde três variáveis são modificadas e recebem novos valores. Note, porém, que é necessário executar os métodos internos \verb|MyTestSetup._calc_secondary_vars()| e \verb|MyAirfoil._calc_grid()| para recalcular as variáveis secundárias depois das variáveis primárias serem redefinidas.

\begin{lstlisting}[caption={Método alternativo para carregar variáveis},label={lst:AltLoadingVars}]
import amiet_tools as AmT

# Inicializando os objetos para armazenar variaveis
MyTestSetup = AmT.loadTestSetup()
MyAirfoil = AmT.loadAirfoilGeom()

# Definir novos valores
MyTestSetup.Ux = 120
MyTestSetup.turb_intensity = 0.05
MyAirfoil.b = 0.15

# IMPORTANTE: recalcular as variaveis secundarias em ambos objetos
MyTestSetup._calc_secondary_vars()
MyAirfoil._calc_grid()
\end{lstlisting}

Desta forma, as diversas variáveis relacionadas ao setup experimental e ao aerofólio utilizado podem ser importadas e combinadas; por exemplo, pode-se simular dois aerofólios diferentes dentro das mesmas condições experimentais. As variáveis podem ser acessadas como atributos dos objetos criados (por exemplo, \verb|DARP2016Setup.c0| para acessar a velocidade do som), ou ainda podem ser exportadas diretamente para o ``namespace'' atual utilizando o método \verb|export_values| (pertencente às duas classes), conforme indicado na listagem \ref{lst:ExportingVars}. Desta forma, as variáveis agora existem tanto como atributos dos objetos criados quanto como variáveis independentes dentro do namespace atual, facilitando certos cálculos.

\begin{lstlisting}[caption={Exportando as variáveis},label={lst:ExportingVars}]
# export variables to current namespace
(c0, rho0, p_ref, Ux, turb_intensity, length_scale, z_sl, Mach, beta,
flow_param, dipole_axis) = DARP2016Setup.export_values()

(b, d, Nx, Ny, XYZ_airfoil, dx, dy) = DARP2016Airfoil.export_values()
\end{lstlisting}

Finalmente, as variáveis relacionadas à frequência temporal $f_0$ (em Hz) são criadas dentro do próprio código, dependendo do interesse do usuário. Neste exemplo inicial, vamos definir uma única frequência $f_0$ a partir da frequência normalizada pela corda $k_0 c = 5$. Para tal, definimos o valor desejado de $f_0$ (a partir da frequência não-dimensional $k_0 c$), e passamos esta variável e o objeto \verb|DARP2016TestSetup| como argumentos para a função \verb|AmT.FrequencyVars|, resultando no objeto aqui chamado \verb|FreqVars|. Este objeto é uma instância da classe \verb|FrequencyVariables|, definida no pacote AmT; mais informações sobre as classes usadas no pacote AmT podem ser encontradas na Seção \ref{sec:ClassDescription}. Similarmente, os valores aqui calculados podem ser exportados para o namespace atual através do método \verb|FreqVars.export_values|.

\begin{lstlisting}[caption={Criando variáveis relacionadas à frequência},label={lst:FrequencyVars}]
# frequency of operation
kc = 5                          # chordwise normalised frequency = k0*(2*b)
f0 = kc*c0/(2*np.pi*(2*b))      # approx 1.8 kHz

FreqVars = AmT.FrequencyVars(f0, DARP2016Setup)
(k0, Kx, Ky_crit) = FreqVars.export_values()
\end{lstlisting}

% *-*-*-*-*-*-*-*-*-*-*-*-*-*-*-*-*-*-*-*-*-*-*-*-*-*-*-*-*-*-*-*-*-*-*-*-*-*-
\section{Exemplos}

\subsection{Exemplo 1: Interação de Rajada Única}

Como primeiro exemplo, vamos usar o pacote AmT para calcular a distribuição do ``salto'' de pressão $\Delta p(x_s, y_s)$ na superfície do aerofólio em resposta a uma única rajada turbulenta, e as funções de diretividade do campo acústico nas direções da corda e da envergadura.



% *-*-*-*-*-*-*-*-*-*-*-*-*-*-*-*-*-*-*-*-*-*-*-*-*-*-*-*-*-*-*-*-*-*-*-*-*-*-
\section{Pseudocódigo}

O pseudocódigo usado para calcular o espectro cruzado $S_{\Delta p \Delta p'}$ da pressão de superfície do aerofólio e/ou o espectro cruzado $S_{pp'}$ do ruído radiado é mostrado abaixo, retirado de \cite{Casagrande_etal2020}. A maior parte das instruções mostradas abaixo possuem funções Python equivalentes ou métodos associados no pacote AmT, e não precisam ser implementadas pelo usuário final. Para mais detalhes sobre o uso destas funções, veja a documentação e os scripts de exemplo no pacote.

\begin{algorithm}
	\caption{Pseudocódigo para o modelo de predição de ruído de interação turbulência-aerofólio do pacote \texttt{amiet\_tools}}
	\label{alg:amiet_pseudocode}
	\begin{algorithmic}[1] % Define line numbers
		\State Definir as variáveis do ambiente de teste: $c_0, \rho_0, p_{ref}, U_x, \overline{w^2}, \Lambda, z_{sl}, M_x, \beta$
		\State Definir as variáveis da geometria do aerofólio: $b, d, N_x, N_y, \Delta x_s, \Delta y_s$
		\State Calcular a amostragem da corda do aerofólio $x_s[n], \ n \in [1, \ldots, N_x]$
		\State Calcular a amostragem da envergadura do aerofólio $y_s[n], \ n \in [1, \ldots, N_y]$
		\State Definir as coordenadas dos observadores $\mathbf{r}_m = (x_m, y_m, z_m), \ m \in [1, \ldots, M]$
		
		\If{usando correção de camada cisalhante}
		\ForAll{pares $(m, n)$ de pontos observador-aerofólio $(\mathbf{r}_m|\mathbf{r}_s)$}
		\State Calcular os pontos de cruzamento da camada cisalhante $\mathbf{r}_l(m, n)$ e tempos de propagação $\tau_{sm}(m, n)$
		\EndFor
		\EndIf
		
		\For{cada frequência $f \in [f_{min}, \ldots, f_{max}]$}
		\State Calcular $k_0, \kappa_\chi, k_\psi^{crit}$
		\State Calcular valores de números de onda de rajada $k_\psi \in [-k_\psi^{max}, k_\psi^{max}]$
		\State Calcular intervalo de amostragem dos números de onda de rajada $\Delta k_\psi$
		\State Calcular espectro de energia dos números de onda de rajada $\Phi_{ww}(k_\psi)$
		\State Calcular matriz $\partial G(\mathbf{r}_m|\mathbf{r}_s)/\partial z_s$ para todos os pares $(m, n)$ de pontos observador-aerofólio; 
		\State Inicializar $S_{\Delta p \Delta p'} \gets 0$
		\State Inicializar $S_{p p'} \gets 0$
		\For{cada rajada $k_\psi \in [-k_\psi^{max}, k_\psi^{max}]$}
		\State $w(k_\psi) \gets \sqrt{\Phi_{ww}(k_\psi)}$
		\State Calcular $\Delta p(x_s, y_s, k_\psi)$ usando $w(k_\psi)$ 
		\State $S_{\Delta p \Delta p'}(\mathbf{r}_s, \mathbf{r}_s') \gets S_{\Delta p \Delta p'}(\mathbf{r}_s, \mathbf{r}_s') + \left[ \Delta p(x_s, y_s, k_\psi) \cdot \Delta p^*(x_s', y_s', k_\psi) \right] \cdot U_x \cdot \Delta k_\psi$ \label{alg:crossproduct}
		\EndFor
		\State $S_{pp'}(\mathbf{r}_m, \mathbf{r}_{m'}) \gets 4 \pi \left[ S_{\Delta p \Delta p'} \cdot (\Delta x_s \cdot \Delta y_s) \cdot (\Delta x_s' \cdot \Delta y_s') \right] \cdot \left[ \partial G(\mathbf{r}_m|\mathbf{r}_s)/\partial z_s \right] \cdot \left[ \partial G(\mathbf{r}_{m'}|\mathbf{r}_s')/\partial z_s \right]$
		\EndFor
	\end{algorithmic}
\end{algorithm}

\clearpage
\newpage

% *-*-*-*-*-*-*-*-*-*-*-*-*-*-*-*-*-*-*-*-*-*-*-*-*-*-*-*-*-*-*-*-*-*-*-*-*-*-
\section{Descrição das classes}
\label{sec:ClassDescription}

O pacote \verb|amiet_tools| utiliza classes - i.e. estruturas de abstração usadas em programação orientada a objetos - para armazenar dados relacionados aos diferentes aspectos da simulação. As três classes utilizadas em AmT estão descritas abaixo, seguidas de uma lista de seus atributos:

\begin{itemize}
	\item Classe \verb|TestSetup|:
	\begin{itemize}
		\item \verb|TestSetup.c0|: velocidade do som em um meio em repouso $c_0$ (em m/s);
		\item \verb|TestSetup.rho0|: densidade do ar $\rho_0$ (em kg/m$^3$);
		\item \verb|TestSetup.p_ref|: pressão acústica de referência $p_{ref} = 20 \mu$Pa RMS;
		\item \verb|TestSetup.Ux|: velocidade média do escoamento $U_x$ (em m/s);
		\item \verb|TestSetup.turb_intensity|: intensidade da turbulência $\overline{w^2}/U_x$;
		\item \verb|TestSetup.length_scale|: escala de comprimento de turbulência $\Lambda$ (em m);
		\item \verb|TestSetup.z_sl|: altura da camada cisalhante $z_{sl}$ (em m);
		\item \verb|TestSetup.flow_dir|: caractere único indicando a direção do escoamento (i.e. `\verb|x|' para indicar escoamento na direção $+x$);
		\item \verb|TestSetup.dipole_axis|: caractere único indicando a direção do eixo dos dipolos (i.e. `\verb|z|' para dipolos apontando ``para cima'', na direção $+z$);
		\item \verb|TestSetup.Mach|: número de Mach do escoamento $M_x = U_x/c_0$;
		\item \verb|TestSetup.beta|: fator de Prandtl-Glauert $\beta = \sqrt{1-M_x^2}$.
	\end{itemize}
	
	\item Classe \verb|AirfoilGeom|:
	\begin{itemize}
		\item \verb|AirfoilGeom.b|: semi corda do aerofólio $b = c/2$ (em m);
		\item \verb|AirfoilGeom.d|: semi envergadura do aerofólio $d = L/2$ (em m);
		\item \verb|AirfoilGeom.Nx|: número de amostras sobre a corda $N_x$, amostragem não-uniforme;
		\item \verb|AirfoilGeom.Ny|: número de amostras sobre a envergadura $N_y$, amostragem uniforme;
		\item \verb|AirfoilGeom.dx|: intervalos de amostragem sobre a corda $\Delta x_s$  (calculado internamente)
		\item \verb|AirfoilGeom.dy|: intervalo de amostragem sobre a envergadura $\Delta y_s$  (calculado internamente).
	\end{itemize}
	
	\item Classe \verb|FrequencyVariables|:
	\begin{itemize}
		\item \verb|FrequencyVariables.freq|: frequência $f$ (em Hz);
		\item \verb|FrequencyVariables.k0|: número de onda acústico $k_0 = 2 \pi f/c_0$;
		\item \verb|FrequencyVariables.Kx|: número de onda de rajada na direção da corda $\kappa_\chi = \omega/U_x$;
		\item \verb|FrequencyVariables.Ky_crit|: número de onda crítico de rajada na direção da envergadura $k_\psi^{crit}$.
	\end{itemize}
\end{itemize}

Uma instância de cada classe é criada para armazenar os valores das variáveis relacionados às condições de um experimento e para passá-las às diferentes funções no pacote. Valores numéricos podem ser atribuídos diretamente através de um script em Python, ou podem ser lidos em um arquivo externo através das funções \verb|loadTestSetup| e \verb|loadAirfoilGeom|. 

Uma quarta classe chamada \verb|MicArrayCsmHDF5| também está inclusa no pacote, e permite a leitura e escrita de dados de CSM de arranjos de microfones no formato HDF5 adotado pela comunidade ``Array Methods'' \cite{Bahr_etal2017, Sarradj_etal2017} para compartilhar dados de ``beamforming''.

% *-*-*-*-*-*-*-*-*-*-*-*-*-*-*-*-*-*-*-*-*-*-*-*-*-*-*-*-*-*-*-*-*-*-
\clearpage
\newpage
\label{Bibliography}
\bibliographystyle{unsrtnat} 
\bibliography{../Bibliography} 
\end{document}
